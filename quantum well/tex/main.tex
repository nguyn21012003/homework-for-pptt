%!TeX TXS-program:compile = txs:///pdflatex/[--shell-escape]
\documentclass{article}
\usepackage[utf8]{vietnam}
\usepackage[utf8]{inputenc}
\usepackage{anyfontsize,fontsize}
\changefontsize[13pt]{13pt}
\usepackage{commath}
\usepackage{parskip}
\usepackage{xcolor}
\definecolor{LightGray}{gray}{0.9}
\usepackage{amssymb}
\usepackage{slashed}
\usepackage{indentfirst}
\usepackage{pdfpages}
\usepackage{graphicx}
\usepackage{minted}


\usepackage{mathtools}
\usepackage{amsfonts}
\usepackage{amsmath,systeme,bbold}
\usepackage[thinc]{esdiff}
\usepackage{hyperref}

\usepackage{lipsum}
\usepackage{fancyhdr}
%footnote
\pagestyle{fancy}
\renewcommand{\headrulewidth}{0pt}%
\fancyhf{}%
\fancyfoot[L]{Vật lý Lý thuyết}%
\fancyfoot[C]{\hspace{4cm} \thepage}%

\newcommand{\image}[2]{
	\begin{center}
		\includegraphics[width=1\textwidth]{#1}
		\textbf{Figure. 1:} #2
		\label{fig:#1}
	\end{center}
}

\usepackage{geometry}
\geometry{
	a4paper,
	total={170mm,257mm},
	left=20mm,
	top=20mm,
}

\renewcommand{\baselinestretch}{2.0}


\title{\Huge{GIẾNG THẾ VUÔNG HỮU HẠN}}

\hypersetup{
	colorlinks=true,
	linkcolor=red,
	filecolor=magenta,      
	urlcolor=cyan,
	pdftitle={},
	pdfpagemode=FullScreen,
}

\urlstyle{same}

\begin{document}
	\setlength{\parindent}{20pt}
	\newpage
	\author{TRẦN KHÔI NGUYÊN \\LÊ QUỐC DUY \\ PHẠM NGUYỄN THÀNH ĐẠT \\NGUYỄN LÊ KHẢI HOÀN \\ LÊ THƯỢNG PHƯƠNG ANH}
	\maketitle
	\newpage
	\section{Lý thuyết}
	\subsection{Giếng thế vuông hữu hạn}
	Xét giếng thế vuông hữu hạn:
	\begin{align}
		V(x) = 
		\begin{cases}
			-V_0 &, \quad \quad -a  \leq x \leq a ,\\ 
			0    &, \quad \quad \abs{x} > a,
		\end{cases}
	\end{align}
	trong đó $V_0$ là hằng số dương.
	
	Xét trường hợp trạng thái liên kết. Trong vùng $x<a$, $V(x) = 0$(nằm ngoài giếng), phương trình Schr\"{o}dinger có dạng:
	\begin{align}
		-\dfrac{\hbar}{2m}\dfrac{d^2\psi}{dx^2} = E \psi, \quad \text{hoặc} \; \dfrac{d^2\psi}{dx^2} = \kappa^2 \psi,
	\end{align}
	với $\kappa \equiv \dfrac{\sqrt{-2mE}}{\hbar}$ là số thực dương. Nghiệm tổng quát cho $\psi(x) = A e^{-\kappa x} + B e^{\kappa x}$. Số hạng thứ nhất bị triệt tiêu khi $x \rightarrow -\infty$. Nên nghiệm tổng quát cho $\psi$ đơn giản thành:
	\begin{align}
		\psi(x) = B e^{\kappa x} , \quad(x< -a)
	\end{align}
	Trong vùng $-a < x < a$, phương trình Schr\"{o}dinger có dạng:
	\begin{align}
		-\dfrac{\hbar}{2m}\dfrac{d^2\psi}{dx^2} - V_0 \psi = E \psi, \quad \text{hoặc} \; \dfrac{d^2\psi}{dx^2} = -l^2 \psi,
	\end{align}
	với $l \equiv \dfrac{\sqrt{2m(E + V_0)}}{\hbar}$. Mặc dù $E$ mang giá trị âm, nhưng đối với trạng thái liên kết, $E$ vẫn phải lớn hơn $-V_0$, và $l$ cũng phải là số thực dương. Nghiệm tổng quát cho $\psi$ trong vùng $-a < x < a$ là:
	\begin{align}
		\psi(x) = C \sin (lx) + D \cos (lx) , \quad \quad ( -a < x < a),
	\end{align}
	$C,D$ là các hằng số bất kì. Nghiệm tổng quát cho $\psi$ tại vùng $x>a$: $\psi(x) = F e^{-\kappa x} + G e^{\kappa x}$, nhưng số hạng cuối bị triệt tiêu khi $x \rightarrow \infty$. Vậy hàm sóng đơn giản thành:
	\begin{align}
		\psi(x) = F e^{-\kappa x}, \quad \quad (x>a).
	\end{align}
	
	Hàm sóng cho 3 vùng có dạng:
	\begin{align}
		\psi(x) =
		\begin{cases}
			F e^{-\kappa x} ,&\quad \quad (x>a),\\
			D \cos (lx) , &\quad \quad (0<x<a),\\
			\psi(-x), &\quad \quad (x<0).
		\end{cases}
	\end{align}
	Nhờ vào tính liên tục của hàm sóng $\psi$, tại $x = a$:
	\begin{align}
		F e^{-\kappa a} = D \cos (la),
	\end{align}
	và tính liên tục của đạo hàm hàm sóng $\psi$:
	\begin{align}
		-\kappa F e^{-\kappa a} = -l D \sin (la)
	\end{align}
	Chia 2 vế phương trình (8) , (9) ta được:
	\begin{align}
		\kappa = l \tan (la)
	\end{align}
	
	Khi lấy $l^2 + \kappa ^2$, thành phần năng lượng bị triệt tiêu. Nên ta có:
	\begin{align}
		l^2 + \kappa^2 = \frac{2mV_0}{\hbar^2}
	\end{align}
	Vì $\kappa ,l$ là các hàm theo $E$, để giải tìm $E$ chúng ta sẽ cần đưa ra một số kí hiệu:
	\begin{align}
		z \equiv la, \quad \quad z_0 \equiv \frac{a}{\hbar} \sqrt{2mV_0} 
	\end{align}
	Phân tích thứ nguyên cho $z_0$
	\begin{align}
		[z_0] &= \left[ \dfrac{m}{eV\times s} (kg \times eV \times 1.6e-19)^{1/2}  \right] \nonumber \\
		& = \left[ \dfrac{m}{eV\times s} (kg \times J)^{1/2}  \right] \nonumber \\
		& = \left[ \dfrac{m}{eV\times s} (kg \times \dfrac{kg \times m^2}{s^2})^{1/2}  \right] \nonumber\\
		& = \left[ \dfrac{kg\times m^2}{eV\times s^2}\right]  = \left[\dfrac{J}{eV}\right] = constant,
	\end{align}
	ta có thể thấy $z_0$ là đại lượng không có thứ nguyên.
	
	Từ phương trình (11):
	\begin{align}
		(l^2 + \kappa^2)a = a\frac{2mV_0}{\hbar^2} = z_0^2 \nonumber\\
		\Rightarrow \kappa a = \sqrt{z_0^2 - z^2} \\
		\Rightarrow \tan z = \sqrt{\left( \dfrac{z0}{z} \right)^2 - 1}.
	\end{align}
	Phương trình (15) có thể giải số để tìm ra tất cả nghiệm có thể bằng các điểm cắt nhau giữa $RHS$ và $LHS$.
	\begin{itemize}
		\item \textbf{Giếng sâu, rộng}. Với $a,V_0 \gg$, điều này đồng nghĩa với $z_0 \gg$. 
		\item \textbf{Giếng nông, hẹp}. Với $a,V_0 \ll$, điều này đồng nghĩa với $z_0 \ll$. $z_0$ càng giảm ta càng có ít nghiệm cho trạng thái liên kết. Với $z_0 < \dfrac{\pi}{2}$, ta chỉ còn lại một nghiệm cho trạng thái liên kết. Mặc cho $z_0$ có nhỏ bao nhiêu, ta luôn tìm được một nghiệm cho trạng thái liên, mặc dù rất yếu.
	\end{itemize}
	
	
	
	
	
	
	
	
	
	
	
	
	
	\section{Giải số}
	giải số như nào v.v
	
	\newpage
	\section{Phương pháp}
	\subsection{Phương pháp Bisection}
	\begin{minted}[bgcolor=LightGray]{python}
def bisection(f, a, b, eps, N, z0):
for i in range(N):
	c = (a + b) / 2
	if abs(f(c, z0)) < eps:
		break
	if f(a, z0) * f(c, z0) < 0:
		b = c
	elif f(c, z0) * f(b, z0) < 0:
		a = c
	else:
		break
return c
	\end{minted}
	\subsection{Phương pháp Newton Raphson}
	\begin{minted}[bgcolor=LightGray]{python}
def newton(f, p0, eps, N, z0):
p0_n = p0
for n in range(0, N + 1):
p0_n = p0_n - f(p0_n, z0) / df(p0_n, z0)
if df(p0_n, z0) == 0:
	break
if abs(f(p0_n, z0)) < eps:
	break
return p0_n
	\end{minted}
	\subsection{Phương pháp Secant}
	\begin{minted}[bgcolor=LightGray]{python}
def secant(f, p0, p1, eps, N, z0):
p0_n = p0
p1_n = p1
for n in range(N + 1):
	df = (f(p1_n, z0) - f(p0_n, z0)) / (p1_n - p0_n)
	p_n = p0_n - f(p0_n, z0) / df
if f(p0_n, z0) * f(p_n, z0) < 0:
	p0_n = p0_n
	p1_n = p_n
elif f(p1_n, z0) * f(p_n, z0) < 0:
	p0_n = p_n
	p1_n = p1_n
if abs(f(p_n, z0)) < eps:
	break
if abs(p1_n - p0_n) < eps:
	break
if f(p_n, z0) == 0:
	break
return p_n
	\end{minted}
	\section{Kết quả}
	\image{Final.pdf}{a) Nghiệm của phương trình $\tan z = \sqrt{\left( \frac{z0}{z} \right)^2 - 1}$. b) Hàm sóng ứng với các mức năng lượng khi $a = 10nm, V_0 = 1eV$ .}
	

	
\end{document}