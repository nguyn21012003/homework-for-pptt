%!TeX TXS-program:compile = txs:///pdflatex/[--shell-escape]
\documentclass{article}
\usepackage[utf8]{vietnam}
\usepackage[utf8]{inputenc}
\usepackage{anyfontsize,fontsize}
\changefontsize[13pt]{13pt}
\usepackage{commath}
\usepackage{parskip}
\usepackage{xcolor}
\definecolor{LightGray}{gray}{0.9}
\usepackage{amssymb}
\usepackage{slashed}
\usepackage{indentfirst}
\usepackage{pdfpages}
\usepackage{graphicx}
\usepackage{minted}


\usepackage{mathtools}
\usepackage{amsfonts}
\usepackage{amsmath,systeme,bbold}
\usepackage[thinc]{esdiff}
\usepackage{hyperref}

\usepackage{lipsum}
\usepackage{fancyhdr}
%footnote
\pagestyle{fancy}
\renewcommand{\headrulewidth}{0pt}%
\fancyhf{}%
\fancyfoot[L]{Vật lý Lý thuyết}%
\fancyfoot[C]{\hspace{4cm} \thepage}%



\usepackage{geometry}
\geometry{
	a4paper,
	total={170mm,257mm},
	left=20mm,
	top=20mm,
}

\renewcommand{\baselinestretch}{2.0}


\title{\Huge{GIẾNG THẾ VUÔNG HỮU HẠN}}

\hypersetup{
	colorlinks=true,
	linkcolor=red,
	filecolor=magenta,      
	urlcolor=cyan,
	pdftitle={},
	pdfpagemode=FullScreen,
}

\urlstyle{same}

\begin{document}
	\setlength{\parindent}{20pt}
	\newpage
	\author{TRẦN KHÔI NGUYÊN \\LÊ QUỐC DUY \\ PHẠM NGUYỄN THÀNH ĐẠT \\NGUYỄN LÊ KHẢI HOÀN \\ LÊ THƯỢNG PHƯƠNG ANH}
	\maketitle
	\newpage
	\section{Lý thuyết}
	\subsection{Potential}
	
	\section{Giải số}
	giải số như nào v.v
	\section{Phương pháp}
	\subsection{Phương pháp Bisection}
	\begin{minted}[bgcolor=LightGray]{python}
		import numpy as np
		from numpy import tan, sqrt
		from math import pi
		
		
		def function(z, z0, N):
			f = tan(z) - sqrt((z0 / z) ** 2 - 1)
		return f
		
		
		def bisection(f, a, b, N, eps):
		a = float(a)
		b = float(b)
		if a > b:
		a = b
		b = a
		
		na = np.zeros(N)
		nb = np.zeros(N)
		nc = np.zeros(N)
		na[0] = a
		nb[0] = b
		
		count = 0
		
		for i in range(N):
		if i + 1 < N:
		nc[i] = (na[i] + nb[i]) / 2
		
		if f(nc[i]) == 0:
		break
		
		if abs(f(nc[i])) < eps:
		break
		
		if f(na[i]) * f(nc[i]) < 0:
		nb[i + 1] = nc[i]
		na[i + 1] = na[i]
		count += 1
		
		elif f(nc[i]) * f(nb[i]) < 0:
		nb[i + 1] = nb[i]
		na[i + 1] = nc[i]
		count += 1
		
		return (na, nb, nc), count
		
		
		def main():
		N = 100
		hbar = 1.05457182e-34
		m = 9.31e-31
		V0 = 32 * hbar**2 / (m * a**2)
		a = 1
		z0 = a / hbar * sqrt(2 * m * V0)
		function(z0, hbar, N)
		
		
		if __name__ == "__main__":
		main()
		
	\end{minted}
	\subsection{Phương pháp Newton Raphson}
	\subsection{Phương pháp Secant}
	\section{Kết quả}
	
\end{document}