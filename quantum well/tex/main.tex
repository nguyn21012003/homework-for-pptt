\documentclass{article}
\usepackage[utf8]{vietnam}
\usepackage[utf8]{inputenc}
\usepackage{anyfontsize,fontsize}
\changefontsize[13pt]{13pt}
\usepackage{commath}
\usepackage{parskip}
\usepackage{xcolor}
\definecolor{LightGray}{gray}{0.9}
\usepackage{amssymb}
\usepackage{slashed}
\usepackage{indentfirst}
\usepackage{pdfpages}
\usepackage{graphicx}



\usepackage{mathtools}
\usepackage{amsfonts}
\usepackage{amsmath,systeme,bbold}
\usepackage[thinc]{esdiff}
\usepackage{hyperref}

\usepackage{lipsum}
\usepackage{fancyhdr}
%footnote
\pagestyle{fancy}
\renewcommand{\headrulewidth}{0pt}%
\fancyhf{}%
\fancyfoot[L]{Vật lý Lý thuyết}%
\fancyfoot[C]{\hspace{4cm} \thepage}%

\newcommand{\image}[3]{
	\begin{center}
		\includegraphics[width=1\textwidth]{#1}
		\textbf{Figure. #2:} #3
		\label{fig:#1}
	\end{center}
}	


\usepackage{geometry}
\geometry{
	a4paper,
	total={170mm,257mm},
	left=20mm,
	top=20mm,
}

\renewcommand{\baselinestretch}{2.0}
\newcommand{\at}[2]{\bigg\rvert_{#1}^{#2} }

\title{\Huge{GIẾNG THẾ VUÔNG HỮU HẠN}}

\hypersetup{
	colorlinks=true,
	linkcolor=black,
	filecolor=magenta,      
	urlcolor=cyan,
	pdftitle={},
	pdfpagemode=FullScreen,
}

\urlstyle{same}

\begin{document}
	\setlength{\parindent}{20pt}
	\newpage
	\author{TRẦN KHÔI NGUYÊN \\LÊ QUỐC DUY \\ PHẠM NGUYỄN THÀNH ĐẠT \\NGUYỄN LÊ KHẢI HOÀN \\ LÊ THƯỢNG PHƯƠNG ANH}
	\maketitle
	
	\newpage
	\tableofcontents
	\newpage
	\section{Lý thuyết}
	\subsection{Bài toán vật lý}
	Xét giếng thế vuông hữu hạn:
	\begin{align}
		V(x) = 
		\begin{cases}
			-V_0 &, \quad \quad -a  \leq x \leq a ,\\ 
			0    &, \quad \quad \abs{x} > a,
		\end{cases}
	\end{align}
	trong đó $V_0$ là hằng số dương.
	
	
	
	Xét trường hợp trạng thái liên kết. Trong vùng $x<a$, $V(x) = 0$(nằm ngoài giếng), phương trình Schr\"{o}dinger có dạng:
	\begin{align}
		-\dfrac{\hbar}{2m}\dfrac{d^2\psi}{dx^2} = E \psi, \quad \text{hoặc} \; \dfrac{d^2\psi}{dx^2} = \kappa^2 \psi,
	\end{align}
	với $\kappa \equiv \dfrac{\sqrt{-2mE}}{\hbar}$ là số thực dương. Nghiệm tổng quát cho $\psi(x) = A e^{-\kappa x} + B e^{\kappa x}$. Số hạng thứ nhất bị triệt tiêu khi $x \rightarrow -\infty$. Nên nghiệm tổng quát cho $\psi$ đơn giản thành:
	\begin{align}
		\psi(x) = B e^{\kappa x} , \quad(x< -a)
	\end{align}
	Trong vùng $-a < x < a$, phương trình Schr\"{o}dinger có dạng:
	\begin{align}
		-\dfrac{\hbar}{2m}\dfrac{d^2\psi}{dx^2} - V_0 \psi = E \psi, \quad \text{hoặc} \; \dfrac{d^2\psi}{dx^2} = -l^2 \psi,
	\end{align}
	với $l \equiv \dfrac{\sqrt{2m(E + V_0)}}{\hbar}$. Mặc dù $E$ mang giá trị âm, nhưng đối với trạng thái liên kết, $E$ vẫn phải lớn hơn $-V_0$, và $l$ cũng phải là số thực dương. Nghiệm tổng quát cho $\psi$ trong vùng $-a < x < a$ là:
	\begin{align}
		\psi(x) = C \sin (lx) + D \cos (lx) , \quad \quad ( -a < x < a),
	\end{align}
	$C,D$ là các hằng số bất kì. Nghiệm tổng quát cho $\psi$ tại vùng $x>a$: $\psi(x) = F e^{-\kappa x} + G e^{\kappa x}$, nhưng số hạng cuối bị triệt tiêu khi $x \rightarrow \infty$. Vậy hàm sóng đơn giản thành:
	\begin{align}
		\psi(x) = F e^{-\kappa x}, \quad \quad (x>a).
	\end{align}
	\subsection{Tính chẵn lẻ của hàm sóng}
	Thế năng ta chọn là hằng số nên xem là hàm chẵn $V_0(-x)=V_0(x)$. Khi đó có thể chứng minh rằng, hàm sóng sẽ là hàm chẵn hoặc hoặc lẻ $\psi (-x) = \pm \psi (x)$. Trong bài toán này, ta sẽ chọn nghiệm là hàm chẵn. Do đó ta dùng sự đối xứng để hiệu chỉnh hàm sóng:
	\begin{itemize}
		\item Bỏ số hạng $C \sin{lx}$ của hàm sóng trong khoảng $(-a,a)$.
		\item $\psi(x < -a) = \psi( x > a )=F e^{ \kappa x }$
	\end{itemize} 
	Hàm sóng cho 3 vùng có dạng:
	\begin{align}
		\psi(x) =
		\begin{cases}
			F e^{-\kappa x} ,&\quad \quad (x>a),\\
			D \cos (lx) , &\quad \quad (0<x<a),\\
			\psi(-x), &\quad \quad (x<0).
		\end{cases}
	\end{align}
	Nhờ vào tính liên tục của hàm sóng $\psi$, tại $x = a$:
	\begin{align}
		F e^{-\kappa a} = D \cos (la),
	\end{align}
	và tính liên tục của đạo hàm hàm sóng $\psi$:
	\begin{align}
		-\kappa F e^{-\kappa a} = -l D \sin (la).
	\end{align}
	Chia 2 vế phương trình (8) , (9) ta được:
	\begin{align}
		\kappa = l \tan (la).
	\end{align}
	
	Khi lấy $l^2 + \kappa ^2$, thành phần năng lượng bị triệt tiêu. Nên ta có:
	\begin{align}
		l^2 + \kappa^2 = \frac{2mV_0}{\hbar^2}
	\end{align}
	Vì $\kappa ,l$ là các hàm theo $E$, để giải tìm $E$ chúng ta sẽ cần đưa ra một số kí hiệu:
	\begin{align}
		z \equiv la, \quad \quad z_0 \equiv \frac{a}{\hbar} \sqrt{2mV_0} .
	\end{align}
	Phân tích thứ nguyên cho $z_0$
	\begin{align}
		[z_0] &= \left[ \dfrac{m}{eV\times s} (kg \times eV \times 1.6e-19)^{1/2}  \right] \nonumber \\
		& = \left[ \dfrac{m}{eV\times s} (kg \times J)^{1/2}  \right] \nonumber \\
		& = \left[ \dfrac{m}{eV\times s} (kg \times \dfrac{kg \times m^2}{s^2})^{1/2}  \right] \nonumber\\
		& = \left[ \dfrac{kg\times m^2}{eV\times s^2}\right]  = \left[\dfrac{J}{eV}\right] = constant,
	\end{align}
	ta có thể thấy $z_0$ là đại lượng không có thứ nguyên.
	
	Từ phương trình (11):
	\begin{align}
		(l^2 + \kappa^2)a^2 &= a^2\frac{2mV_0}{\hbar^2} = z_0^2 \nonumber\\
		\Rightarrow \kappa a &= \sqrt{z_0^2 - z^2} \\
		\Rightarrow \tan z &= \sqrt{\left( \dfrac{z_0}{z} \right)^2 - 1}.
	\end{align}
	Phương trình (15) có thể giải số để tìm ra tất cả nghiệm có thể bằng các điểm cắt nhau giữa $RHS$ và $LHS$.
	\begin{itemize}
		\item \textbf{Trường hợp giếng sâu rộng:} $a_0$ và $V_0$ càng lớn nên $z_0$ càng lớn, đồ thị vế phải sẽ giao với đồ thị của vế trái càng nhiều điểm. Với điều kiện xác định của căn thức thì ta sẽ sẽ có tập xác định của z là $0 < z < z_0$. Cộng thêm tính tuần hoàn của $\tan{z}$ thì $z \in (0,\frac{\pi}{2}) \cup (\frac{\pi}{2} + k\pi , \frac{\pi}{2} + (k+1)\pi) (k= 0,1,2,3...)$. Với mỗi điểm cắt sẽ ứng với một mức năng lượng ($z$ là hàm của năng lượng). Và với mỗi điểm đó sẽ ứng với một hàm sóng.
		\item \textbf{Trường hợp giếng nông hẹp:} Là khi $z_0$ rất nhỏ. Số nghiệm của phương trình sẽ ít vì hàm căn thức rất hẹp nên sẽ cắt hàm $\tan{z}$ ít điểm hơn. Nhưng dù $z_0$ rất nhỏ tiến tới 0 thì vẫn tồn tại ít nhất một điểm cắt, nghĩa là vẫn sẽ luôn tồn tại một trạng thái cơ bản của hệ rất yếu.
	\end{itemize}
	
	\newpage
	
	\section{Phương pháp giải}
	\subsection{Phương pháp hình học}
	
	Một cách nhanh nhất để giải và tìm nghiệm cho $\tan z = \sqrt{\left( \dfrac{z_0}{z} \right)^2 - 1}$ đó là vẽ $LHS$ và $RHS$ trên cùng một đồ thị. Tại những điểm cắt nhau trên đồ thị, đó là nghiệm của bài toán.
	\image{tan.pdf}{1}{Nghiệm của phương trình $\tan z = \sqrt{\left( \frac{z_0}{z} \right)^2 - 1}$}
	Mặc dù cho $z_0$ rất bé, ta luôn tìm được một nghiệm chẵn. Không nhất thiết phải cần có nghiệm lẻ.
	\subsection{Giải số}
	
	Sử dụng phương pháp hình học đem lại cho ta kết quả nhanh, tiện lợi. Nhưng bên cạnh đó cũng có thể giải phương trình $\tan z = \sqrt{\left( \frac{z_0}{z} \right)^2 - 1}$ bằng thuật toán truy hồi sử dụng các phương pháp Chia đôi khoảng cách, phương pháp Newton - Raphson, phương pháp Secant.
	
	\begin{align}
		y(z) = \tan z - \sqrt{\left( \frac{z_0}{z} \right)^2 - 1} = 0
	\end{align}
	Mục tiêu bây giờ là tìm ra điểm giao nhau bằng 0 của hàm này và đó sẽ là nghiệm ta cần tìm. Nhưng ở số hạng thứ nhất hàm $\tan$ sẽ bị phân kì khi $z = \pi/2 + k \pi$, và ở số hạng thứ hai bị phân kì khi $z = 0$ hoặc $z_0 < z$. Để giải quyết vấn đề này, chúng tôi đưa ra một phương pháp giải đó là giải số $y(z)$ trong khoảng lân cận $0+ \epsilon$ $\rightarrow $ $\dfrac{\pi}{2} - \epsilon$ và cứ thế tăng một $k\pi$ với $k = 0,1,2,3...$
	
	\subsubsection{Thuật toán các phuơng pháp}
	\image{Bisection.png}{2}{Thuật toán Bisection}
	\image{Secant.png}{3}{Thuật toán Secant}
	\image{Newton.png}{4}{Thuật toán Newton - Raphson}
	
	\newpage
	
	\section{Kết quả}
	\image{Final.pdf}{5}{a) Nghiệm của phương trình $\tan z = \sqrt{\left( \frac{z_0}{z} \right)^2 - 1}$. b) Hàm sóng ứng với các mức năng lượng khi $a = 10nm, V_0 = 1eV$. c) Các mức năng lượng ứng với mỗi $z_0$}
	
	\image{wavefunctions.pdf}{6}{Hàm sóng ở các trạng thái dừng. Với $a_0 = 10nm, V_0 = 1eV $}

	Khi giải số cho phương trình $\tan z = \sqrt{\left( \frac{z_0}{z} \right)^2 - 1}$, chúng ta đã sử dụng các thông số như sau: $V_0 = 1eV, a = 10 nm,m = 0.067 \times m_e$ với số lần lặp $N= 10000$. \\
	Với $z_1$ ứng với mức năng lượng $E_1 = - 0.9894098922642349 eV$ , $z_2$ ứng với mức năng lượng $E_2 = - 0.9063791214121885 eV$, $z_3$ ứng với mức năng lượng là $E_3 = - 0.7600531455945667 eV$.

	\clearpage
	\section{Appendix}
	\subsection{Hệ số chuẩn hóa}
	\begin{align*}
		1 &= 2 \int_{0}^{\infty} \abs{\psi(x)}^2 \, dx \\
		&= 2 \left[ \int_{0}^{a} \abs{D}^{2} \cos^{2}(lx) \, dx + \int_{a}^{\infty} \abs{F}^{2} e^{-2\kappa x} \, dx \right] \\
		&= 2 \left[ \int_{0}^{a} \abs{D}^{2} \left(\frac{1}{2} + \frac{1}{2} \cos(2lx)\right) \, dx + \int_{a}^{\infty} \abs{F}^{2} e^{-2\kappa x} \, dx \right]\\
		&= 2\left[\abs{D}^{2}\left(\frac{x}{2} + \frac{\sin(2lx)}{4l}\right)\at{0}{a} + \abs{F}^{2} \frac{e^{-2 \kappa x}}{-2\kappa} \at{a}{\infty} \right]\\
		&= 2 \left[ \abs{D}^2 \left(\frac{a}{2} + \frac{\sin(2la)}{4l}\right) + \abs{F}^2 \frac{e^{-2\kappa a}}{2\kappa}\right]
	\end{align*}
	Ta xét điều kiện biên liên tục tại $x = a$:
	\begin{align}
		&F e^{-\kappa a} = D \cos(la) \nonumber\\
		\Rightarrow &F = D e^{ka} \cos(la)
	\end{align}
	Xét đạo hàm theo biến $x$ của $\psi$:
	\begin{align}
		&\frac{d(F e^{-kx})}{dx} \at{x=a^+}{} = \frac{d(D \cos(lx))}{dx} \at{x=a^-}{} \nonumber \\
		&\Rightarrow -\kappa F e^{-\kappa a}  = - l D \sin(la) \nonumber\\
		&\Rightarrow \kappa = l \tan(la).
	\end{align}
	Thay (1) và (3) vào ta được:
	\begin{align*}
		1 &= 2 \left[ \abs{D}^2 \left(\dfrac{a}{2} + \dfrac{\sin(2la)}{4l}\right) + 	e^{2\kappa a} \cos^{2}(la) \frac{e^{-2\kappa a}}{2k}\right] \\
		&= 2 \left[ \abs{D}^2 \left(\dfrac{a}{2} + \dfrac{\sin(2la)}{4l}\right) + \abs{D}^2 	\dfrac{\cos^{2}(la)}{2\kappa}\right] \\
		&= \abs{D}^2 \left[a + \dfrac{\sin(2la)}{2l} + \dfrac{\cos^{2}(la)}{\kappa}\right] \\
		&= \abs{D}^2 \left(a + \dfrac{\sin(2la)}{2l} + \dfrac{\cos^{2}(la)}{l 	\tan(la)}\right) \\
		&= \abs{D}^2 \left[ a + \dfrac{\sin(la)\cos(la)}{l} + \frac{\cos^{3}(la)}{l 	\sin(la)}\right] \\
		&= \abs{D}^2 \left[a + \dfrac{\cos(la)}{l \sin(la)} \left(\sin^2(la) + 	\cos^2(la)\right)\right] \\
		&= \abs{D}^2 \left[a + \dfrac{1}{l \tan(la)}\right] \\
		1 &= \abs{D}^2 \left(a + 1/\kappa \right) \\
		\Rightarrow D &= \sqrt{\dfrac{1}{a + 1/\kappa }} \\
		\Rightarrow F &= D e^{\kappa a} \cos(la) \\
		&= \sqrt{\dfrac{1}{a + 1/\kappa }} e^{\kappa a} \cos(la)\kappa
	\end{align*}
	
	
	
\end{document}